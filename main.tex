\documentclass[12pt]{article}
\renewcommand{\thesection}{\Roman{section}} 
\renewcommand{\thesubsection}{\thesection.\Roman{subsection}}
%\usepackage[tocindentauto]{tocstyle}
%\usetocstyle{KOMAlike} %the previous line resets it
%\usepackage{natbib}
\usepackage{biblatex}
\addbibresource[]{ref.bib}
\usepackage{url}
\usepackage[utf8]{inputenc}
\usepackage{amsmath}
\usepackage{graphicx}
\usepackage{graphviz}
\usepackage[T1]{fontenc}
\graphicspath{{images/}}
\usepackage{parskip}
\usepackage{fancyhdr}
\usepackage{hyperref}
\usepackage{parskip}
\usepackage{hologo}
\usepackage{listings}
\usepackage{titlesec, blindtext, color}
\usepackage{titling}
\usepackage{tcolorbox}
\usepackage[hmargin=1in,vmargin=1in]{geometry}
\usepackage{float}
\usepackage{tikz}
\usepackage{appendix}
\usepackage{listings} % For code importing
\usepackage{xcolor} % for setting colors
\usepackage{svg}
\usepackage{tocloft}
\renewcommand{\cftsecleader}{\cftdotfill{\cftdotsep}}

\input{arduinoLanguage.tex}

\hypersetup{
	colorlinks=true,
	linkcolor=blue,
	urlcolor=cyan,
}

\lstdefinestyle{customc}{
  belowcaptionskip=1\baselineskip,
  breaklines=true,
  frame=L,
  xleftmargin=\parindent,
  language=C,
  showstringspaces=false,
  basicstyle=\footnotesize\ttfamily,
  keywordstyle=\bfseries\color{green!40!black},
  commentstyle=\itshape\color{purple!40!black},
  identifierstyle=\color{blue},
  stringstyle=\color{orange},
 }

 \lstset{ %
  backgroundcolor=\color{white},   % choose the background color; you must add \usepackage{color} or \usepackage{xcolor}
  basicstyle=\footnotesize,        % the size of the fonts that are used for the code
  breakatwhitespace=false,         % sets if automatic breaks should only happen at whitespace
  breaklines=true,                 % sets automatic line breaking
  captionpos=b,                    % sets the caption-position to bottom
  commentstyle=\color{commentsColor}\textit,    % comment style
  deletekeywords={...},            % if you want to delete keywords from the given language
  escapeinside={\%*}{*)},          % if you want to add LaTeX within your code
  extendedchars=true,              % lets you use non-ASCII characters; for 8-bits encodings only, does not work with UTF-8
  frame=tb,	                   	   % adds a frame around the code
  keepspaces=true,                 % keeps spaces in text, useful for keeping indentation of code (possibly needs columns=flexible)
  keywordstyle=\color{keywordsColor}\bfseries,       % keyword style
  language=Python,                 % the language of the code (can be overrided per snippet)
  otherkeywords={*,...},           % if you want to add more keywords to the set
  numbers=left,                    % where to put the line-numbers; possible values are (none, left, right)
  numbersep=8pt,                   % how far the line-numbers are from the code
  numberstyle=\tiny\color{commentsColor}, % the style that is used for the line-numbers
  rulecolor=\color{black},         % if not set, the frame-color may be changed on line-breaks within not-black text (e.g. comments (green here))
  showspaces=false,                % show spaces everywhere adding particular underscores; it overrides 'showstringspaces'
  showstringspaces=false,          % underline spaces within strings only
  showtabs=false,                  % show tabs within strings adding particular underscores
  stepnumber=1,                    % the step between two line-numbers. If it's 1, each line will be numbered
  stringstyle=\color{stringColor}, % string literal style
  tabsize=2,	                   % sets default tabsize to 2 spaces
  title=\lstname,                  % show the filename of files included with \lstinputlisting; also try caption instead of title
  columns=fixed                    % Using fixed column width (for e.g. nice alignment)
}

\lstdefinestyle{customasm}{
  belowcaptionskip=1\baselineskip,
  frame=L,
  xleftmargin=\parindent,
  language=[x86masm]Assembler,
  basicstyle=\footnotesize\ttfamily,
  commentstyle=\itshape\color{purple!40!black},
}

\lstset{escapechar=@,style=customc}

%\makeatletter
%\let\thetitle\@title

%\let\thedate\@date
%\makeatother

%\pagestyle{fancy}
%\fancyhf{}
%\rhead{\theauthor}
%\lhead{\thetitle}
%\cfoot{\thepage}

\begin{document}
\title{Project Proposal}
%%%%%%%%%%%%%%%%%%%%%%%%%%%%%%%%%%%%%%%%%%%%%%%%%%%%%%%%%%%%%%%%%%%%%%%%%%%%%%%%%%%%%%%%%

\begin{titlepage}
	\centering
    \vspace*{0.5 cm}
    \includegraphics[scale = 0.11]{isu_seal.png}\\[1.0 cm]	% University Logo
    \textsc{\LARGE IOWA STATE UNIVERSITY}\\[2.0 cm]
    \textsc{\large AEROSPACE ENGINEERING DEPARTMENT}\\[0.2 cm]
    \textsc{\large Computational Techniques for Aerospace Design}\\[0.2 cm]
	\textsc{\Large AERE 361}\\[0.5 cm]				% Course Code
	\textsc{\Large Project Proposal}\\[0.2 cm]
	\textsc{\Large TEAM NAME HERE}\\[0.2 cm]
	\rule{\linewidth}{0.2 mm} \\[0.4 cm]
	%{ \huge \bfseries \thetitle}\\
	
	
	\begin{minipage}{0.8\textwidth}
		
			\begin{flushleft} 
			\emph{Team Member Names :} \\
			Last Name, First Name\linebreak
			Last Name, First Name\linebreak
			Last Name, First Name\linebreak
			Last Name, First Name\linebreak
			Last Name, First Name\linebreak
			Last Name, First Name\linebreak
			
		\end{flushleft}
	\end{minipage}\\[2 cm]
	
	\vfill
	
\end{titlepage}

%%%%%%%%%%%%%%%%%%%%%%%%%%%%%%%%%%%%%%%%%%%%%%%%%%%%%%%%%%%%%%%%%%%%%%%%%%%%%%%%%%%%%%%%%
%\maketitle
\tableofcontents
\pagebreak
%%%%%%%%%%%%%%%%%%%%%%%%%%%%%%%%%%%%%%%%%%%%%%%%%%%%%%%%%%%%%%%%%%%%%%%%%%%%%%%%%%%%%%%%%

\section{ABSTRACT}
The abstract is a summary of your proposal. In general, your abstract should have enough information so that if I was to copy and paste your abstract into a web site, people would get the general idea of what your proposal is about. It should not go into any heavy detail, just the basics of what your project is about. The who, the what, and the why. You should keep your abstract to 200-400 words. Use this to ``hook in'' your reader.

\section{INTRODUCTION}
While the abstract and introduction may seem like it is similar, remember that your abstract should have enough information to stand on its own. The introduction is really the actual start to your proposal. Here you should introduce the project, the people involved and give a short introduction to the why you are doing this. This should be 1-3 paragraphs.

\section{FEATURES}
Your Features section must include a listing of at least three key features that makes your project unique. Each item needs to be backed up with a description of what it will do and why. A listing of just three items is not enough, you need to describe what those features are and why your group feels they are needed. For that reason your features should have a paragraph for each key item that describes what that key feature is. A key feature should be something that is significant to your project. For example, a key feature an autopilot system is the ability to be able to set an altitude and the autopilot will automatically set the airspeed. That is a significant feature that has a large impact on that system.


% Below is an example of inserting an image.  Not that LaTex
% will determine the best location for the image.  Make sure
% you replace this image with yours and place a proper caption.
% You can use the \label{name} to name the figure and then reference
% it from your writeup and LaTeX will automatically give it the correct
% number. 
\begin{figure}[!t]
\centering
\includegraphics[width=4.5in]{cpx01.jpg}
\caption{This is the Circuit Playground Express}
\label{fig:cpx}
\end{figure}

\section{PROBLEM STATEMENT}
College life is something that all of us here at Iowa State adjust to. The freedoms you gain hand in hand with the responsibilities you must bear. The freedom to make your own schedule comes with the responsiblity to manage your own time. The freedom of choice means you need to choose wisely. With mental health issues on the rise, we can view this as a major stressor that will be placed on the entire student body. Acording to The National College Health Assessment performed in fall of 2021, almost 45 percent of students are impacted by stress, and another 35.59 percent deal with anxiety realated issues. Therefore we should look into combatting these ailments, and one of the best ways listed by the Anxiety and Depression Association of America is exercise, or really any physical activity.

So what is keeping student's from being active? Perhaps its because they don't have enough space to be active? Unlikely, with all the gyms, open areas, and clubs that offer dozzens of opertunities to be active. But, as far as personal space goes, students can be quite limited while living in dorms. So the size of student's personal lives has been impacted without thinking about how they will likely share this space with at least one roommate. Typical stress and anxiety reducing activities may need to be hidden or at least less noticable as to maintain some kind of privacy for the students. Otherwise, they may not be comfortable to do said anxiety combatting activities for fear of being seen as abnormal. So, we need to find some way to keep students somewhat physically active while constrained to their smaller spaces.

\section{PROBLEM SOLUTION}
Here go over your approach to your solution and what your solution is. You must include at least one image that shows your concept. This image can be a sketch or drawing or some pictures that show your concept. Make sure you reference the image(s) like this - Figure \ref{fig:cpx}. Finally, make sure you replace the stock image I included. You should also reference any sources you had from your problem statement as well.

You must also include a table that lists all the parts that you wish to have. As announced in class, you will have the parts listed in Table \ref{table:parts_list}. We have plenty of two additional parts. Those are a conductive adhesive strip and a neopixel strip. I do have some other parts, such as arcade buttons and some additional sensors. You can certainly ask for something, and I will see what I can do. Change the table below to reflect the parts you are requesting.

\begin{table}[ht]
  \caption{Parts available for teams}
  \label{table:parts_list}
  \begin{center}
  \begin{tabular}{|p{3in}|c|}
  
  \hline
  Part description & Qty\\
  \hline
  \hline
  Adafruit Circuit Playground Express & 1 \\
  \hline
  AAA Battery Holder & 1 \\
  \hline
  USB Cable & 1 \\
  \hline
  \end{tabular}
  \end{center}
  \end{table}

Finally, you can also include any pseudo code or any code snippets you have gathered so far.  This is not required, but if you found some starter code or came up with some ideas for the code, put it here. If you want to embed code into \LaTeX, you can use the example below on how to do this in \LaTeX.

\begin{lstlisting}[language=Arduino]
#include <Adafruit_CircuitPlayground.h>

void setup() {
  CircuitPlayground.begin();
}

void loop() {
  CircuitPlayground.clearPixels();

  delay(500);

  CircuitPlayground.setPixelColor(0, 255,   0,   0);
  CircuitPlayground.setPixelColor(1, 128, 128,   0);
  CircuitPlayground.setPixelColor(2,   0, 255,   0);
  CircuitPlayground.setPixelColor(3,   0, 128, 128);
  CircuitPlayground.setPixelColor(4,   0,   0, 255);
  
  CircuitPlayground.setPixelColor(5, 0xFF0000);
  CircuitPlayground.setPixelColor(6, 0x808000);
  CircuitPlayground.setPixelColor(7, 0x00FF00);
  CircuitPlayground.setPixelColor(8, 0x008080);
  CircuitPlayground.setPixelColor(9, 0x0000FF);
 
  delay(5000);
}
\end{lstlisting}

\section{CONCLUSION}
Finally, wrap up your proposal. This only needs to be one or two paragraphs, but it should conclude with what you plan to do and the why and how. Yes, this may seem repetitive, but that is intentional. Do not forget to update your references as those will appear below in a seperate page.

\newpage
%\section{References}
\printbibliography[heading=subbibintoc]
%\bibliographystyle{plain}
%\bibliography{ref}
“Campus Health and Wellness Data.” Student Wellness, Iowa State University, 2021, https://www.studentwellness.iastate.edu/campus-health-and-wellness-data/. 

\end{document}
