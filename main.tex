\documentclass[12pt]{article}
\renewcommand{\thesection}{\Roman{section}} 
\renewcommand{\thesubsection}{\thesection.\Roman{subsection}}
%\usepackage[tocindentauto]{tocstyle}
%\usetocstyle{KOMAlike} %the previous line resets it
%\usepackage{natbib}
\usepackage{biblatex}
\addbibresource[]{ref.bib}
\usepackage{url}
\usepackage[utf8]{inputenc}
\usepackage{amsmath}
\usepackage{graphicx}
\usepackage{graphviz}
\usepackage[T1]{fontenc}
\graphicspath{{images/}}
\usepackage{parskip}
\usepackage{fancyhdr}
\usepackage{hyperref}
\usepackage{parskip}
\usepackage{hologo}
\usepackage{listings}
\usepackage{titlesec, blindtext, color}
\usepackage{titling}
\usepackage{tcolorbox}
\usepackage[hmargin=1in,vmargin=1in]{geometry}
\usepackage{float}
\usepackage{tikz}
\usepackage{appendix}
\usepackage{listings} % For code importing
\usepackage{xcolor} % for setting colors
\usepackage{svg}
\usepackage{tocloft}
\renewcommand{\cftsecleader}{\cftdotfill{\cftdotsep}}

\input{arduinoLanguage.tex}

\hypersetup{
	colorlinks=true,
	linkcolor=blue,
	urlcolor=cyan,
}

\lstdefinestyle{customc}{
  belowcaptionskip=1\baselineskip,
  breaklines=true,
  frame=L,
  xleftmargin=\parindent,
  language=C,
  showstringspaces=false,
  basicstyle=\footnotesize\ttfamily,
  keywordstyle=\bfseries\color{green!40!black},
  commentstyle=\itshape\color{purple!40!black},
  identifierstyle=\color{blue},
  stringstyle=\color{orange},
 }

 \lstset{ %
  backgroundcolor=\color{white},   % choose the background color; you must add \usepackage{color} or \usepackage{xcolor}
  basicstyle=\footnotesize,        % the size of the fonts that are used for the code
  breakatwhitespace=false,         % sets if automatic breaks should only happen at whitespace
  breaklines=true,                 % sets automatic line breaking
  captionpos=b,                    % sets the caption-position to bottom
  commentstyle=\color{commentsColor}\textit,    % comment style
  deletekeywords={...},            % if you want to delete keywords from the given language
  escapeinside={\%*}{*)},          % if you want to add LaTeX within your code
  extendedchars=true,              % lets you use non-ASCII characters; for 8-bits encodings only, does not work with UTF-8
  frame=tb,	                   	   % adds a frame around the code
  keepspaces=true,                 % keeps spaces in text, useful for keeping indentation of code (possibly needs columns=flexible)
  keywordstyle=\color{keywordsColor}\bfseries,       % keyword style
  language=Python,                 % the language of the code (can be overrided per snippet)
  otherkeywords={*,...},           % if you want to add more keywords to the set
  numbers=left,                    % where to put the line-numbers; possible values are (none, left, right)
  numbersep=8pt,                   % how far the line-numbers are from the code
  numberstyle=\tiny\color{commentsColor}, % the style that is used for the line-numbers
  rulecolor=\color{black},         % if not set, the frame-color may be changed on line-breaks within not-black text (e.g. comments (green here))
  showspaces=false,                % show spaces everywhere adding particular underscores; it overrides 'showstringspaces'
  showstringspaces=false,          % underline spaces within strings only
  showtabs=false,                  % show tabs within strings adding particular underscores
  stepnumber=1,                    % the step between two line-numbers. If it's 1, each line will be numbered
  stringstyle=\color{stringColor}, % string literal style
  tabsize=2,	                   % sets default tabsize to 2 spaces
  title=\lstname,                  % show the filename of files included with \lstinputlisting; also try caption instead of title
  columns=fixed                    % Using fixed column width (for e.g. nice alignment)
}

\lstdefinestyle{customasm}{
  belowcaptionskip=1\baselineskip,
  frame=L,
  xleftmargin=\parindent,
  language=[x86masm]Assembler,
  basicstyle=\footnotesize\ttfamily,
  commentstyle=\itshape\color{purple!40!black},
}

\lstset{escapechar=@,style=customc}

%\makeatletter
%\let\thetitle\@title

%\let\thedate\@date
%\makeatother

%\pagestyle{fancy}
%\fancyhf{}
%\rhead{\theauthor}
%\lhead{\thetitle}
%\cfoot{\thepage}

\begin{document}
\title{Project Proposal}
%%%%%%%%%%%%%%%%%%%%%%%%%%%%%%%%%%%%%%%%%%%%%%%%%%%%%%%%%%%%%%%%%%%%%%%%%%%%%%%%%%%%%%%%%

\begin{titlepage}
	\centering
    \vspace*{0.5 cm}
    \includegraphics[scale = 0.11]{isu_seal.png}\\[1.0 cm]	% University Logo
    \textsc{\LARGE IOWA STATE UNIVERSITY}\\[2.0 cm]
    \textsc{\large AEROSPACE ENGINEERING DEPARTMENT}\\[0.2 cm]
    \textsc{\large Computational Techniques for Aerospace Design}\\[0.2 cm]
	\textsc{\Large AERE 361}\\[0.5 cm]				% Course Code
	\textsc{\Large Project Proposal}\\[0.2 cm]
	\textsc{\Large Team Daedalus}\\[0.2 cm]
	\rule{\linewidth}{0.2 mm} \\[0.4 cm]
	%{ \huge \bfseries \thetitle}\\
	
	
	\begin{minipage}{0.8\textwidth}
		
			\begin{flushleft} 
			Finrock, Andrew\linebreak
			Herrera, Ricardo\linebreak
			Miller, Otis\linebreak
			White, James\linebreak
			Zhang, Yolanda\linebreak
			
		\end{flushleft}
	\end{minipage}\\[2 cm]
	
	\vfill
	
\end{titlepage}

%%%%%%%%%%%%%%%%%%%%%%%%%%%%%%%%%%%%%%%%%%%%%%%%%%%%%%%%%%%%%%%%%%%%%%%%%%%%%%%%%%%%%%%%%
%\maketitle
\tableofcontents
\pagebreak
%%%%%%%%%%%%%%%%%%%%%%%%%%%%%%%%%%%%%%%%%%%%%%%%%%%%%%%%%%%%%%%%%%%%%%%%%%%%%%%%%%%%%%%%%

\section{ABSTRACT}
As it is, the living spaces of modern college students are not that big. Dorms and apartments alike are created with the intent of housing as many students as possible in as small of a space as possible. This creates an obstacle for students when it comes to the types of recreational activities they are able to pursue in these small spaces. Inspired by the basketball-shooting games found in arcades, we have decided to create a scaled-down version that will allow college students to practice their basketball-shooting skills in a confined environment. This game will also help foster friendly competition among students and will provide a chance for them to gather together. 

Our basketball-shooting game will display many of the same features found on the commercial games found in arcades. To accomplish this, we will be using the Circuit Playground Express circuit board and its array of sensors. Firstly, we will be design the physical court using inexpensive materials such as cardboard and plastic. We will then use the CPX board's light sensor, which will be mounted on the backboard, to detect when a ball is passed through the hoop. We will also create a program to display the time remaining on the game as well as the current score. Upon a successful shot, the user's score will increment, and this will repeat until the timer runs out. We will also utilize the LED lights on the CPX board to create a visual effect whenever a shot is made. With all of these features implemented, we will have created a fun game that college students can utilize in their small living spaces.

\section{INTRODUCTION}
In this project, our team will use our programming skills learned in AERE 361 along with the Adafruit Circuit Playground Express to create an arcade-like mini basketball game. Our team, Team Daedalus consists of members, Andrew Finfrock, Ricardo Herrera, Otis Miller, James White, and Yolanda Zhang. We are all upperclassmen majoring in aerospace engineering at Iowa State University and we are currently taking the Advanced Programming for Aerospace class, AERE 361. 

For this project, we wanted to create a fun, portable, simple game that could be played in small spaces yet still be entertaining. We wanted a game that could be used for stress relief as a quick study break, or as a friendly competition when hanging out with friends. With this criteria in mind, we landed on the idea to make an arcade-like mini basketball game, perfect for this project with the added bonus of improving one's hand-eye coordination. 

\section{FEATURES}
Your Features section must include a listing of at least three key features that makes your project unique. Each item needs to be backed up with a description of what it will do and why. A listing of just three items is not enough, you need to describe what those features are and why your group feels they are needed. For that reason your features should have a paragraph for each key item that describes what that key feature is. A key feature should be something that is significant to your project. For example, a key feature an autopilot system is the ability to be able to set an altitude and the autopilot will automatically set the airspeed. That is a significant feature that has a large impact on that system.


% Below is an example of inserting an image.  Not that LaTex
% will determine the best location for the image.  Make sure
% you replace this image with yours and place a proper caption.
% You can use the \label{name} to name the figure and then reference
% it from your writeup and LaTeX will automatically give it the correct
% number. 
\begin{figure}[!t]
\centering
\includegraphics[width=4.5in]{cpx01.jpg}
\caption{This is the Circuit Playground Express}
\label{fig:cpx}
\end{figure}

\section{PROBLEM STATEMENT}
Here you will go into more detail on what problem you hope to solve or address.  You should discuss what the problem is and why it is important to solve it. In this section, you need to be clear on what the problem is, so do not think of this as a ``light'' section. It helps to define your project.

Your team needs to do some research into the problem at hand. Becuase of that, you should have around two to three references that you are pulling from. There are lots of places you can find references from including the ISU library and Google Scholar. I would also suggest looking at Adafruit's website, as you may find inspiration or looking to improve something already there. Remember to cite your sources though. If you find something online, that can often be citation.

When you create your ``ref.bib'' file, don't forget to follow the standards for a BiBTex file. Certain things like webistes requires certain keywords for it to render properly. There are lots of sources online to help with this and many places like the ISU Library and Google Scholar can also generate text that is compatible with a BiBTex file. Once you have your Bib file ready, don't forget to cite your citations in your proposal like this \cite{einstein} or this \cite{dirac}.

\section{PROBLEM SOLUTION}
Here go over your approach to your solution and what your solution is. You must include at least one image that shows your concept. This image can be a sketch or drawing or some pictures that show your concept. Make sure you reference the image(s) like this - Figure \ref{fig:cpx}. Finally, make sure you replace the stock image I included. You should also reference any sources you had from your problem statement as well.

You must also include a table that lists all the parts that you wish to have. As announced in class, you will have the parts listed in Table \ref{table:parts_list}. We have plenty of two additional parts. Those are a conductive adhesive strip and a neopixel strip. I do have some other parts, such as arcade buttons and some additional sensors. You can certainly ask for something, and I will see what I can do. Change the table below to reflect the parts you are requesting.

\begin{table}[ht]
  \caption{Parts available for teams}
  \label{table:parts_list}
  \begin{center}
  \begin{tabular}{|p{3in}|c|}
  
  \hline
  Part description & Qty\\
  \hline
  \hline
  Adafruit Circuit Playground Express & 1 \\
  \hline
  AAA Battery Holder & 1 \\
  \hline
  USB Cable & 1 \\
  \hline
  \end{tabular}
  \end{center}
  \end{table}

Finally, you can also include any pseudo code or any code snippets you have gathered so far.  This is not required, but if you found some starter code or came up with some ideas for the code, put it here. If you want to embed code into \LaTeX, you can use the example below on how to do this in \LaTeX.

\begin{lstlisting}[language=Arduino]
#include <Adafruit_CircuitPlayground.h>

void setup() {
  CircuitPlayground.begin();
}

void loop() {
  CircuitPlayground.clearPixels();

  delay(500);

  CircuitPlayground.setPixelColor(0, 255,   0,   0);
  CircuitPlayground.setPixelColor(1, 128, 128,   0);
  CircuitPlayground.setPixelColor(2,   0, 255,   0);
  CircuitPlayground.setPixelColor(3,   0, 128, 128);
  CircuitPlayground.setPixelColor(4,   0,   0, 255);
  
  CircuitPlayground.setPixelColor(5, 0xFF0000);
  CircuitPlayground.setPixelColor(6, 0x808000);
  CircuitPlayground.setPixelColor(7, 0x00FF00);
  CircuitPlayground.setPixelColor(8, 0x008080);
  CircuitPlayground.setPixelColor(9, 0x0000FF);
 
  delay(5000);
}
\end{lstlisting}

\section{CONCLUSION}
With small living spaces, it can be difficult for college students to find a form of friendly completion that doesn't lie within video games. Our proposition to this challenge is to create a scaled-down basketball-shooting game using a CPX circuit board. The multitude of sensors and LED lights on the build will allow us to create a functional game that can provide students with entertainment and competition among their peers.

\newpage
%\section{References}
\printbibliography[heading=subbibintoc]
%\bibliographystyle{plain}
%\bibliography{ref}

\end{document}
